
\documentclass{article}
\begin{document}
	  \textbf{\underline{HISTORICAL CONTEXT OF FIVE (5)PROGRAMMING LAGUAGE}}
	  \section{MAPLE}
	  \begin{itemize}
	  	\item \underline{FOUNDER:}
	  	\end{itemize}
  	Waterloo Maple (Maple soft)
  	\begin{itemize}
  		\item \textbf{\underline{HISTORY OF MAPLE}}
  		
  	\end{itemize}
  \begin{itemize}
  	\item Maple offical release date was in 1982.It was developed by Waterloo Maple (Maple Soft).It came about when researchers at university of waterloo wanted to buy a powerful computer strong enough to take on the lisp-based computer algebra system\underline{macsyma.}So instead they developed \textbf{maple} which would run on low cost computers and  could also be portable. These researchers began writing maple from \textbf{BCPL}  . It was first released in 1982.  Then over time Maple had an exponential growth.  Maple is technical developed to cover areas such as data processing.
  	
  	
  \end{itemize}
\underline{\textbf{ADVANTAGE OF MAPLE}}
 It provides an intellectual environment in solving technical problems due to the fact that it can carry out symbolic, numeric and graphical computations. It is a very powerful language even though it was written in C, Java , Maple.
 \begin{itemize}
 	\item But no one uses maple again because it is slow, instead people use mathematica.
 	\item Maple is more or less a mathematical software
 	\item Examples of applications:
 	Fractal leaf generator, Knight’s Tour, The SEIR model with births and deaths. Etc
 	\item Maple IDE-comes with features like
 	Powerful Maple code editor, automatic indenting, source code validation which makes it easy to use.
 	\item \textbf{IDE}(Integrated Development Environment) just enables programmers to incorporate the different aspects of writing a computer  program.
 	\item Related programs – C , Java
 	
 	
 \end{itemize}
	  	
 \end{document}