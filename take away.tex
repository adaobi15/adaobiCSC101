\documentclass{article}

\begin{document}
	\author{ADA THE PRINCESS}
	\title{Cardano's Formula for Cubic Equations}
	\maketitle

	\begin{center}
		\textbf{Abstract}
	\end{center}
	Gerolamo Cardano was born in Pavia 1504 as the illegitimate child of a jurist. He attended the University of Padua and became a physician in the town of Sacco, after being rejected by his home Europe, having treated the Pope. He was also an astrologer and an avid gambler, to which he wrote the Book on Games of chance , which was the first serious treatise on the mathematics of probalitity. \cite{cardanoformula}
	
	\section{ Introduction to Cardano's Formula}
	Cardano's formula for solution of cubic equations for an equation like;
			\begin{equation}
			x^3 + a1x^2 + a2x + a3 = 0
		\end{equation}
		the parameters Q, R , S and T can be computed thus,
		
		\begin{equation}
			Q= \frac{3a_{2} - a{1}^2}{a}
		\end{equation}
		\begin{equation}
			\frac{R=9a_{1}a_{2}-27a_{3}-2a_{1}^3}{54}	
		\end{equation}
		\begin{equation}
			S=3\sqrt{R +\sqrt{-Q^3 + R^2}}
		\end{equation}
		\begin{equation}
			T=\sqrt{R-\sqrt{Q^3 + R^2}}
		\end{equation}
		
		to give the roots
		\begin{equation}
			x_{1}= S + T -\frac{1}{3}a_{1}
		\end{equation}
		\begin{equation}
			x_{2} =\frac{-(S +T)}{2} -\frac{a_{1}}{3} + i\frac{\sqrt{3}(s-T)}{2}
		\end{equation}
		\begin{equation}
			x_{2} =\frac{-(S +T)}{2} -\frac{a_{1}}{3} + i\frac{\sqrt{3}(s-T)}{2}
		\end{equation}
		\subsection{Some examples}
		
		\begin{itemize}
			\item	\begin{equation}
				x^3 - 3x^2 + 4 =0
			\end{equation}
			\item\begin{equation}
				2x^3 + 6x^2 + 1 =0
			\end{equation}
		\end{itemize}
	


\bibliography{take away.bib}
\bibliographystyle{ieeetr}
\end{document}